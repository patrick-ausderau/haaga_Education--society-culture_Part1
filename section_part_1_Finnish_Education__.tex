\section{part 1 - Finnish Education}

Ministry of education would like to raise the level of  education among the population \cite[pp.9-10]{edu_2012} and build stronger links between education and working life \cite[pp.13-18]{edu_2012}.	
If I compare with the Swiss education system, very similar to Finnish system but where apprenticeship equals vocational institutions.
After school, teenagers aged around 15-16 can start to work in a company, usually 3 days a week where they get practical training and get a salary.
Then, for two days a week, they go to a vocational school where they receive more practical training and theory related to their work and also general knowledge.
As seen in Kyrö \cite[pp. 3-5]{kyro_2012}, the amount of upper secondary education is higher in Switzerland and also the unemployment rate in the age group 15-29 is one of the lowest of the OECD.

Still regarding the building of stronger link between education and working life, in Metropolia UAS, this is done through work placement.  
During the third year of the curriculum, the students will work in a company to develop their skill based on what they learned at school. 
This contact with real working life often offers a real job to the student and sometimes also a subject for their bachelor thesis.

Overall, the Finnish educational system impress me with good results in PISA and achievements such as ``background does not determine educational performance'' ~\cite[p.14]{kyro_2012}. 
I also like the idea of equal access to education with many upper secondary and ternary education free of charge.
From my computer science professional background, I am also pleased that the ministry of education \cite[p.18,48]{edu_2012} promotes the usage of information and communications technologies in education.
But unfortunatelly, at same time reduce the amount if ICT students \cite[Appendix 1]{edu_2012}